% Appendix A

\chapter{Programa principal - PDG} % Main appendix title

\label{AppendixA} % For referencing this appendix elsewhere, use \ref{AppendixA}
\lhead{Appendix A. \emph{Programa principal - PDG}} % This is for the header on each page - perhaps a shortened title


\begin{center}

\begin{table}
\captionof{table}{Argumentos del programa principal} \label{tab:PDGArgs} 
\begin{tabular}{l p{10cm}}

\hline\hline
Argumento & Descripción\\
\hline
-m & Modo de ejecución (algoritmo seleccionado). 1 – Aho Corasick 2 – FSM\\
-h & Imprime los argumentos aceptados por el programa\\
-q & Cantidad de patrones a ser tenidos en cuenta\\
-c & Ruta absoluta en la cual se encuentra el archivo con el texto\\
-p & Ruta en la cual se encuentra el archivo con el conjunto de patrones a ser buscados\\
-s & Tamaño del bloque de lectura del texto (tamaño de páginas). Por defecto se utiliza un blque de 1024 páginas.\\
--printMemory & Imprime los valores RSS y VSZ del proceso (solo para sistemas operativos con procfs).\\
--printPreTime & Imprime el tiempo en segundos que tarda en ejecutarse la creación de la estructura del algoritmo que se está ejecutando\\
--readTime & Imprime el tiempo en segundos que tarda en ejecutarse la lectura completa del texto
\\
--readBlockTime & Imprime el tiempo en segundos que tarda en ejecutarse un bloque de lectura del texto\\
--postTime & Imprime el tiempo en segundos que tarda en ejecutarse el post procesamiento del algoritmo (aplicable solo para MSW).\\
--totalTime & Imprime el tiempo en segundos que tarda en ejecutar completamente el algoritmo.\\
--finds & Imprime la cantidad de veces que se encontró cada uno de los patrones. \\
--printComments & Imprime comentarios a la salida con el prefijo $\#$.\\
\hline
\end{tabular}
\label{table:PDG}
\end{table}
\end{center}